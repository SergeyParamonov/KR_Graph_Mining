% !TEX root = notes.tex
\section{Conclusion and future work}
\label{sec:conclusion}
Higher order logic is sometimes criticized as being to expressive.
Some problems however, are inherently higher order, oftentimes due to
a large modular structure.
In this paper we used graph mining as an example of an higher order
problem and made a thorough analysis of the problem from a knowledge
representation point of view.
While techniques exist to express these higher order problems in first order logic,
sometimes, explicitly specifying the additional structure HO exhibits
allows systems to perform better.
For example, in the case of graph mining, higher order logic preserves
the local coherence of graphs, and the independence of homomorphisms
for the different examples, a property that a higher order solver can
leverage in order to raise efficiency.
Inspired by this case study, we propose higher-order language
extensions for IDP and propose alternative ways to implement them in the
solver. In particular as shown in Section~\ref{sec:performance}, the use of subsolvers seems promising and will be further explored together with the idea of Benders decomposition~\citep{Benders}.
The performance of the encodings in IDP or ASP can be considered as
the ultimate target.