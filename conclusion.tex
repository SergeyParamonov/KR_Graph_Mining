% !TEX root = notes.tex
\section{Conclusion and future work}\label{sec:conclusion}
\matthias{A conclusion}
Higher order logic is sometimes criticized as being to expressive.
Some problems however, are inherently higher order, oftentimes due to a large modular structure.
While techniques exist to express these higher order problems in first order logic,
sometimes, explicitly specifying the additional structure HO exhibits allows solvers to perform better.
For example, in the case of graph mining, higher order logic preserves the local coherence of graphs, and the independence of homomorphisms for the different examples, a property that a higher order solver can leverage in order to raise efficiency. 
Concretely, we predict that our subsolver technique will perform like the decomposed model from Section~\ref{sec:performance}, and expect a comparable performance gain over the first order model that uses only the disjoint union technique.