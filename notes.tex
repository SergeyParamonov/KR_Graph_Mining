\documentclass{article}
\usepackage{color}
\usepackage[usenames,dvipsnames]{xcolor}
\usepackage{listings}	
\usepackage{enumerate}
\usepackage{layouts}
\usepackage{times}
\usepackage[numbers]{natbib}
\usepackage{notoccite}
\usepackage{url}
\usepackage{xspace}
\usepackage{tikz}

\author{Authors}
\title{Note on KR and Graph Mining}
\begin{document}
\maketitle
\section{Feature Comparison}

\subsection{IDP} 

\textbf{Pro}:
\begin{itemize}
  \item can model inductive definitions
  \item allows core formulation in a high-level language (NP)
  \item handles aggregates
  \item has support for variety of constraints
\end{itemize}
\textbf{Cons:}
\begin{itemize}
  \item cannot handle negative case $\textit{NP}^\textit{NP}$ complexity
  \item cannot model subgraph isomorphism independence
  \item cannot handle dominance, i.e., when one model is preferred over another 
\end{itemize}

\paragraph{ASP}
Mostly the same but in theory can handle $\textit{NP}^\textit{NP}$, in practice however, it would require encoding tricks and unavoidably lead to the same problem as in IDP -- indexing homomorphism enumeration.

\subsection{proB}
\textbf{Pro}
\begin{itemize}
  \item can model negative case
  \item can model subgraph isomorphism independence
\end{itemize}

\textbf{Cons:}
\begin{itemize}
  \item cannot handle inductive definitions
  \item cannot handle different types of aggregates (? needs to be checked again)
\end{itemize}

the rest of constraints? 




\end{document}
