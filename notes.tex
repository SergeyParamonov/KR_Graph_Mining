\documentclass{article}
\usepackage{color}
\usepackage[usenames,dvipsnames]{xcolor}
\usepackage{listings}	
\usepackage{enumerate}
\usepackage{layouts}
\usepackage{times}
\usepackage[numbers]{natbib}
\usepackage{notoccite}
\usepackage{todonotes}
\usepackage{alltt}
\usepackage{url}
\usepackage{amsmath}
\usepackage{xspace}
\usepackage{tikz}
\usepackage[dvipsnames]{xcolor}

\usepackage{fancyvrb}

% redefine \VerbatimInput
\RecustomVerbatimCommand{\VerbatimInput}{VerbatimInput}%
{fontsize=\footnotesize,
   %
  frame=lines,  % top and bottom rule only
  framesep=2em, % separation between frame and text
  rulecolor=\color{Gray},
      %
  label=\fbox{\color{Black}proB encoding},
  labelposition=topline,
        %
% commandchars=\|\(\), % escape character and argument delimiters for
                              % commands within the verbatim
% commentchar=*        % comment character
}

\author{Authors}
\title{Note on KR and Graph Mining}
\begin{document}
\maketitle

An attempt to model the Graph Mining problem in both IDP as well as ProB makes it clear
that neither language allows us to express the problem to its full extent.
We try to link the shortcomings of each language to the expressiveness of the underlying
logic on which they are built.
%Furthermore, we 

\subsection{IDP}
The IDP language can express \emph{Existential Second Order} problems; problems in which there is an existentially quantified, generally second order, vocabulary of symbols and a first order theory with symbols from that vocabulary.

This restriction to \emph{Existential} Second Order forces us 
%to separate the constraint describing the positive homomorphic property, and the constraint on its number of occurrences.
%This, from a KR point of view, 
%Furthermore, the restriction to ESO requires us 
to use a Skolemization based trick when expressing the homomorphic property.
We introduce a general function \verb|f| that represents the homomorphisms, and make its dependency on a specific goal graph explicit using an additional argument:
\verb|partial f(graph, t_var):node|.
In Second Order Logic, this dependency would follow directly from the order of the separate quantifications.
We can now use this \verb|f| anywhere we would the regular homomorphic function for a specific graph by fixing the goal graph.
Note that this encoding also requires us to make this function \verb|f| partial, as the Graph Mining problem does not require the solution to be homomorphic with \emph{all} goal graphs.
\todo{Of course, other (even uglier) schemes exist to encode this. Should we mention this?}

Furthermore, limiting ourselves to existential second order prohibits us from expressing the constraint negative constraint on homomorphism (No more than $N_{-}$ negative examples are homomorphic) in the same model.
In fact, the negative constraint asserts a property for all candidate homomorphic functions, which would lead to \emph{universal} quantification.
Therefore, our only recourse is to encode the positive constraint and require it to fail when queried.

\subsection{Eventual encoding}
\begin{alltt}
//Homomorphism/2 is a higher-order predicate:
//Edge1 and Edge2 are predicates themselves.
homomorphism(Edge1, Edge2) \(\iff \exists\) f: (\(\forall\) x, y : x \(\neq\) y \(\Rightarrow\) f(x) \(\neq\) f(y)) \(\wedge\)
\end{alltt}


\section{Feature Comparison}

\subsection{IDP} 

\textbf{Pro:}
\begin{itemize}
  \item can model inductive definitions
  \item allows core formulation in a high-level language (NP)
  \item handles aggregates
  \item has support for variety of constraints
\end{itemize}
\textbf{Cons:}
\begin{itemize}
  \item cannot handle negative case $\textit{NP}^\textit{NP}$ complexity
  \item cannot model subgraph isomorphism independence
  \item cannot handle dominance, i.e., when one model is preferred over another 
\end{itemize}

\paragraph{ASP}
Mostly the same but in theory can handle $\textit{NP}^\textit{NP}$, in practice however, it would require encoding tricks and unavoidably lead to the same problem as in IDP -- indexing homomorphism enumeration.

\subsection{proB}
\textbf{Pro:}
\begin{itemize}
  \item can model negative case
  \item can model subgraph isomorphism independence
\end{itemize}
\textbf{Cons:}
\begin{itemize}
  \item cannot handle inductive definitions
  \item cannot handle different types of aggregates (? needs to be checked again)
\end{itemize}

the rest of constraints? 

\section{Code in ProB and IDP}

\VerbatimInput{original_prob_files/PositiveAndNegative.mch}
\pagebreak

\VerbatimInput[label=IDP encoding]{IDPencoding/core_constraints.idp}



\end{document}
