\newpage
\appendix
\section{Code}
\label{app:Code}
This appendix provides the relevant code for the IDP, ASP and ProB systems.
The full IDP code is available at \url{}, while the ASP code is available at \url{} and the ProB code at \url{}. \todo{Sergey: Provide these urls!} 
\begin{lstlisting}[caption=IDP positive constraint, style=model, label=lst:IDPPos]
vocabulary V{
  type node isa nat
  type graphid
  type label

  // Predicates determining the template graph.
  template_edge(node, node) 
  template_label(node):label

  // Predicates describing the positive example graphs
  example_edge(graphid, node, node)
  label(graphid, node):label
  threshold: int

  // Predicates describing the pattern graph
  inpattern(node) // True for the nodes which occur in the pattern
  partial f(graphid, node):node // Represents the homomorphisms with the example graphs
  homowith(graphid) // True for graphs for which f represents a correct homomorphism
  path(node, node) // path(a,b): True if there exists a path from a to b in the pattern
}

theory Positive:V_Pos{
   //The pattern is a connected subgraph of the template: From every node in the pattern, 
   //there exists a path to every other node in the pattern.
   !x,y[node] : x ~= y & inpattern(x) & inpattern(y) => path(x,y).
   {
      path(x,y) <- template_edge(x,y) & inpattern(x) & inpattern(y).
      path(x,y) <- ?z[node] : path(x,z) & path(z,y).
      path(x,y) <- path(y,x).
   }

   //existence of a homomorphic f from the pattern to example graph with graphid gid.
   !gid[graphid] : !x[node]   : homowith(gid) & inpattern(x) <=> ? y[node] : y=f(gid,x).
   !gid[graphid] : !x,y[node] : homowith(gid) & inpattern(x) & inpattern(y) & x~=y => f(gid, x) ~= f(gid,y).
   !gid[graphid] : !x,y[node] : homowith(gid) & inpattern(x) & inpattern(y) & template_edge(x,y) => edge(gid, f(gid,x). f(gid,y)).
   !gid[graphid] : !x[node] : homowith(gid) & inpattern(x) => template_label(x) = label(gid, f(gid,x)).

   // At least N homomorphisms must be found
   #{ gid [graphid] : homowith(graph) } >= threshold.
}
\end{lstlisting}


\lstset{basicstyle=\footnotesize\ttfamily,breaklines=true}
\begin{lstlisting}[caption=ASP positive matching, style=model]
homowith(G) | not_homowith(G) :- positive(G).

1 { f(G,X,V) : node(G,V) } 1 :- positive(G), inpattern(X).

:- homo_with(G), f(G,X,V1), f(G,Y,V2), template_edge(X,Y), 
not edge(G,V1,V2), inpattern(X), inpattern(Y).

positive_count(N) :- N = #count{G:homowith(G)}.

:- positive_count(N), N < 2.
\end{lstlisting}

\begin{lstlisting}[caption=ASP negative matching using saturation technique, label={lst:aspsaturation}, style=model, numbers=left]
map(G,X,v1) | map(G,X,v2) | map(G,X,v3) | map(G,X,v4) :- invar(X), negative(G). @\label{lstline:probspec}@
map(G,X,V) :- saturated(G), t_node(X), node(G,V).

saturated(G) :- t_edge(X,Y), map(G,X,V1), map(G,Y,V2), not edge(G,V1,V2), negative(G), invar(X), invar(Y).
saturated(G) :- map(G,X,V),  map(G,Y,V), X != Y, invar(X), invar(Y). // we cannot map two different template nodes to the same 

neg_homowith(G) :- not saturated(G), negative(G).

negative_count(N) :- N = #count{G:neg_homowith(G)}.
:- negative_count(N), N > 1.

\end{lstlisting}

\begin{lstlisting}[caption=Canonicity template-based check, style=model]
iso(X,x1) | iso(X,x2) | iso(X,x3) | iso(X,x4) :- invar(X).

candidate_var(X) :- iso(_,X).

%not iso!
iso_saturated :- invar(X1), invar(X2), iso(X1,V1), iso(X2,V2),     t_edge(V1,V2), not t_edge(X1,X2). 
iso_saturated :- invar(X1), invar(X2), iso(X1,V1), iso(X2,V2), not t_edge(V1,V2),     t_edge(X1,X2).

iso(X,V) :- invar(X), t_node(V), iso_saturated.

d1(X) :-     invar(X), not candidate_var(X). 
d2(X) :- not invar(X),     candidate_var(X).

not_equal :- d1(X). % check that in fact candidate is different from the pattern itself
not_equal :- d2(X). % check that in fact candidate is different from the pattern itself

iso_saturated :- not not_equal. % should not be completely equal

min_d1(N) :- N = #min{ X: d1(X) }, not iso_saturated.
min_d2(N) :- N = #min{ X: d2(X) }, not iso_saturated.

iso_saturated :- min_d1(N1), min_d2(N2), N1 > N2.
\end{lstlisting}

\begin{lstlisting}[caption=Auxilary predicates -- probably should be moved to appendix, style=model]
%selects subpattern

t_path(X,Y) :- t_edge(X,Y), invar(X), invar(Y).
t_path(X,Y) :- t_edge(X,Z), t_path(Z,Y), invar(X).

:- invar(X), invar(Y), not t_path(X,Y).

0 { invar(X) } 1 :- t_node(X).
% auxilary constraints


edge(G,Y,X) :- edge(G,X,Y).
t_edge(Y,X) :- t_edge(X,Y).
node(G,Y)   :- edge(G,Y,_).
t_node(X)   :- t_edge(X,_).
\end{lstlisting}

\begin{lstlisting}[caption=Canonicity previous solution isomorphism check, style=model]
iso(s1,X,x1) | iso(s1,X,x2) :- invar(X).
iso(s2,X,x2) | iso(s2,X,x3) :- invar(X).

candidate_var(G,X) :- iso(G,_,X).

iso_saturated(G) :- invar(X1), invar(X2), iso(G,X1,V1), iso(G,X2,V2),     t_edge(V1,V2), not t_edge(X1,X2). 
iso_saturated(G) :- invar(X1), invar(X2), iso(G,X1,V1), iso(G,X2,V2), not t_edge(V1,V2),     t_edge(X1,X2). 
iso_saturatea(G) :- not equal(G), iso(G,_,_). 

iso(G,X,V) :- invar(X), t_node(V), iso_saturated(G).

:- not iso_saturated(G), iso(G,_,_).

d1(G,X) :-     invar(X), not candidate_var(G,X), iso(G,_,_).
d2(G,X) :- not invar(X),     candidate_var(G,X).

not_equal(G) :- d1(G,X). % check that in fact candidate is different from the pattern itself
not_equal(G) :- d2(G,X). % check that in fact candidate is different from the pattern itself

equal(G) :- not not_equal(G), iso(G,_,_).

\end{lstlisting}
