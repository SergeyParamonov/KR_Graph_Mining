\newpage
\appendix
\section{Higher Order Simulation Description}
\label{sec:hol_description}
Key dataset characteristics for the experiments, visualized in \textbf{Fig.} \ref{fig:decomposition_fol}, can be found in \textbf{Table} \ref{table:yoshida}.
\begin{table}[thb]
  \caption{Yoshida datasets parameters}
  \label{table:yoshida}
  \begin{tabular}{l c c c c c}
    Name & Number of Graphs & Avg Vertices & Avg Edges & Labels & Possible classes\\
    \hline
    Yoshida & 265 & 20 &  23 &  9 & 2
  \end{tabular}
\end{table}

The experimental setup visualized in \textbf{Fig.} \ref{fig:decomposition_fol} is the following: in both disjoint union and higher order models we mined the patterns from smaller to larger in an iterative fashion. First, we set the pattern length, equal to the number of nodes, to two, then computed graph coverage for the pattern. Based on the coverage we add the pattern as frequent and then compute isomorphic patterns in the template. For each isomorphic graph in the template we add a no-good clause. Once all frequent patterns of the length $n$ are mined, i.e., the solver cannot find any other non-isomorphic patterns of the length $n$, we increase the pattern length to $n+1$, remove all no-goods and repeat the process.

The key difference between the disjoint union model and the higher order simulation model are in the coverage computation. In case of disjoint union model we make a single call to get a pattern such that it is frequent (i.e., matches at least the threshold amount of graphs) and in higher order model we make a single call to get a non-isomorphic candidate graph and then a separate call per graph to find if it is covered or not. If we found that a pattern covers more than a threshold amount of graphs, we stop computing the coverage and add the pattern as frequent.

Both models in the described computations follow the general schema used in the specialized algorithms such as gSpan \cite{gspan}. We have also obtained similar runtime patterns on other standard graph datasets described in \cite{ilp_graph_mining}.


\section{IDP enumeration results}
In this section, we present the experimental results on the general graph mining IDP encoding using theory splitting (that allows incorporating positive, negative examples and other higher order checks in a uniform fashion). We have applied this encoding to Yoshida dataset on positive examples and used the isomorphism check as a negative theory. The results summarized in \textbf{Table} \ref{table:idp:averaged_results}. The results are consistent with the results in \textbf{Fig.} \ref{fig:decomposition_fol} of the more specialized encoding (that uses imperative code around the IDP/ASP calls) based on gSpan schema \cite{gspan}.

\begin{table}[h]
  \centering
  \setlength{\tabcolsep}{2pt}  
  \begin{tabular}{lccccccccccccccc}  Index  &   1 &   2 & 3 &   4 &   5    & 6   & 7   & 8   & 9   & 10  & 11  & 12  & 13 & 14 & 15\\ \hline 
 Runtime  & 148 & 164 & 173 & 199 &   285  & 364 & 401 & 445 & 490 & 533 & 548 & 585 & 591 & 687 & 802
  \end{tabular}
  \caption{Averaged runtimes in seconds for IDP general graph mining encoding with the theory splitting on the Yoshida dataset.} 
  \label{table:idp:averaged_results}
\end{table}

\section{Code}
\label{app:Code}
This appendix provides the relevant code for the IDP, ASP and ProB systems.
The full IDP code is available at\\ \url{https://github.com/SergeyParamonov/KR_Graph_Mining/tree/master/IDPencoding} and at \\\url{https://github.com/SergeyParamonov/LGM},\\ while the ASP code is available at\\ \url{https://github.com/SergeyParamonov/KR_Graph_Mining/tree/master/ASP_experiments}\\ and the ProB code at \\\url{https://github.com/SergeyParamonov/KR_Graph_Mining/tree/master/prob_bench_files}. 
\begin{lstlisting}[caption=IDP positive constraint, style=model, label=lst:IDPPos]
vocabulary V{
  type node isa nat
  type graphid
  type label

  // Predicates determining the template graph.
  template_edge(node, node) 
  template_label(node):label

  // Predicates describing the positive example graphs
  example_edge(graphid, node, node)
  label(graphid, node):label
  threshold: int

  // Predicates describing the pattern graph
  inpattern(node) // True for the nodes which occur in the pattern
  partial f(graphid, node):node // Represents the homomorphisms with the example graphs
  homowith(graphid) // True for graphs for which f represents a correct homomorphism
  path(node, node) // path(a,b): True if there exists a path from a to b in the pattern
}

theory Positive:V_Pos{
   //The pattern is a connected subgraph of the template: From every node in the pattern, 
   //there exists a path to every other node in the pattern.
   !x,y[node] : x ~= y & inpattern(x) & inpattern(y) => path(x,y).
   {
      path(x,y) <- template_edge(x,y) & inpattern(x) & inpattern(y).
      path(x,y) <- ?z[node] : path(x,z) & path(z,y).
      path(x,y) <- path(y,x).
   }

   //existence of a homomorphic f from the pattern to example graph with graphid gid.
   !gid[graphid] : !x[node]   : homowith(gid) & inpattern(x) <=> ? y[node] : y=f(gid,x).
   !gid[graphid] : !x,y[node] : homowith(gid) & inpattern(x) & inpattern(y) & x~=y => f(gid, x) ~= f(gid,y).
   !gid[graphid] : !x,y[node] : homowith(gid) & inpattern(x) & inpattern(y) & template_edge(x,y) => edge(gid, f(gid,x). f(gid,y)).
   !gid[graphid] : !x[node] : homowith(gid) & inpattern(x) => template_label(x) = label(gid, f(gid,x)).

   // At least N homomorphisms must be found
   #{ gid [graphid] : homowith(graph) } >= threshold.
}
\end{lstlisting}


\lstset{basicstyle=\footnotesize\ttfamily,breaklines=true}
\begin{lstlisting}[caption=ASP positive matching, style=model]
0 { homowith(G) } 1 :- positive(G).

1 { f(G,X,V) : node(G,V) } 1 :- positive(G), inpattern(X).

:- used_f(G,X,V1), used_f(G,Y,V2), template_edge(X,Y), not edge(G,V1,V2), inpattern(X), inpattern(Y).
:- used_f(G,X,V),  t_label(X,L), not label(G,V,L), inpattern(X).

used_f(G,X,V) :- homo_with(G), f(G,X,V).
:- used_f(G,X,V), used_f(G,Y,V), X != Y.

positive_count(N) :- N = #count{G:homowith(G)}.

:- positive_count(N), N < 13.
\end{lstlisting}

\begin{lstlisting}[caption=ASP negative matching using saturation technique, label={lst:aspsaturation}, style=model, numbers=left]
map(G,X,v1) | map(G,X,v2) | map(G,X,v3) | map(G,X,v4) :- invar(X), negative(G). @\label{lstline:probspec}@
map(G,X,V) :- saturated(G), t_node(X), node(G,V).

saturated(G) :- t_edge(X,Y), map(G,X,V1), map(G,Y,V2), not edge(G,V1,V2), negative(G), invar(X), invar(Y).
saturated(G) :- map(G,X,V),  map(G,Y,V), X != Y, invar(X), invar(Y). // we cannot map two different template nodes to the same 

neg_homowith(G) :- not saturated(G), negative(G).

negative_count(N) :- N = #count{G:neg_homowith(G)}.
:- negative_count(N), N > 1.

\end{lstlisting}

\begin{lstlisting}[caption=ASP Canonicity template-based check, style=model]
iso(X,x1) | iso(X,x2) | iso(X,x3) | iso(X,x4) :- invar(X).

candidate_var(X) :- iso(_,X).

%not iso!
iso_saturated :- invar(X1), invar(X2), iso(X1,V1), iso(X2,V2),     t_edge(V1,V2), not t_edge(X1,X2). 
iso_saturated :- invar(X1), invar(X2), iso(X1,V1), iso(X2,V2), not t_edge(V1,V2),     t_edge(X1,X2).

iso(X,V) :- invar(X), t_node(V), iso_saturated.

d1(X) :-     invar(X), not candidate_var(X). 
d2(X) :- not invar(X),     candidate_var(X).

not_equal :- d1(X). % check that in fact candidate is different from the pattern itself
not_equal :- d2(X). % check that in fact candidate is different from the pattern itself

iso_saturated :- not not_equal. % should not be completely equal

min_d1(N) :- N = #min{ X: d1(X) }, not iso_saturated.
min_d2(N) :- N = #min{ X: d2(X) }, not iso_saturated.

iso_saturated :- min_d1(N1), min_d2(N2), N1 > N2.
\end{lstlisting}

\begin{lstlisting}[caption=ASP auxilary predicates, style=model]
%selects subpattern

t_path(X,Y) :- t_edge(X,Y), invar(X), invar(Y).
t_path(X,Y) :- t_edge(X,Z), t_path(Z,Y), invar(X).

:- invar(X), invar(Y), not t_path(X,Y).

0 { invar(X) } 1 :- t_node(X).
% auxilary constraints


edge(G,Y,X) :- edge(G,X,Y).
t_edge(Y,X) :- t_edge(X,Y).
node(G,Y)   :- edge(G,Y,_).
t_node(X)   :- t_edge(X,_).
\end{lstlisting}

\begin{lstlisting}[caption=ASP canonicity previous solution isomorphism check, style=model]
iso(s1,X,x1) | iso(s1,X,x2) :- invar(X).
iso(s2,X,x2) | iso(s2,X,x3) :- invar(X).

candidate_var(G,X) :- iso(G,_,X).

iso_saturated(G) :- invar(X1), invar(X2), iso(G,X1,V1), iso(G,X2,V2),     t_edge(V1,V2), not t_edge(X1,X2). 
iso_saturated(G) :- invar(X1), invar(X2), iso(G,X1,V1), iso(G,X2,V2), not t_edge(V1,V2),     t_edge(X1,X2). 
iso_saturatea(G) :- not equal(G), iso(G,_,_). 

iso(G,X,V) :- invar(X), t_node(V), iso_saturated(G).

:- not iso_saturated(G), iso(G,_,_).

d1(G,X) :-     invar(X), not candidate_var(G,X), iso(G,_,_).
d2(G,X) :- not invar(X),     candidate_var(G,X).

not_equal(G) :- d1(G,X). % check that in fact candidate is different from the pattern itself
not_equal(G) :- d2(G,X). % check that in fact candidate is different from the pattern itself

equal(G) :- not not_equal(G), iso(G,_,_).

\end{lstlisting}

\begin{lstlisting}[caption=ProB frequent graph mining model (only positive examples), style=model]
MACHINE GraphMiner_loop_pos
// A version where we add the patterns one at a time; and we only check for positive patterns
// you can use different datasets by changing the INCLUDES section
// This version requires ProB 1.6.0 maybe even 1.6.1-beta to parse LET and IF-THEN-ELSE predicates
INCLUDES
  YoshidaDataset  // currently we can find only two patterns in reasonable time; problem is the reification of the existential quantifier inside the card set comprehension 

DEFINITIONS

  Labels == {6, 8, 7, 17, 16, 9, 53, 35, 15};
  Vertices == 1..36;
  SET_PREF_TIME_OUT == 70000; SET_PREF_MAX_INITIALISATIONS == 2; SET_PREF_MAX_OPERATIONS == 1; 
  CUSTOM_GRAPH_EDGES == last(Patterns)'EDGES;
  CUSTOM_GRAPH_NODES == Vertices; // dom(pattern'EDGES) \/ ran(pattern'EDGES);
  SET_PREF_SMT == TRUE;

  /* The (most general, i.e. ternary) definition of homomorphism. Note ' is the property accessor for records*/
  homomorph_with(FromGraph, iso, ToGraph) == (
    iso : vertices(FromGraph'EDGES) >-> vertices(ToGraph'EDGES) &
    !x.( x:vertices(FromGraph'EDGES) => FromGraph'LABEL(x) = ToGraph'LABEL(iso(x))) &
    // derived constraint for efficiency:
    // !l.(l:ran(ToGraph'LABEL) => card({x|x:Vertices & x|->l : FromGraph'LABEL}) = card({y|y|->l:ToGraph'LABEL})) &
    
    !(x,y).( x|->y : FromGraph'EDGES
         => iso(x)|->iso(y) : ToGraph'EDGES) 
    // does not seem to gain anyting:
    //& !(x,y).( x:dom(ToGraph'LABEL) & y:dom(ToGraph'LABEL) & x|->y /: ToGraph'EDGES 
     //    => !(v,w).(v|->x:iso & w|->y:iso => v|->w /: FromGraph'EDGES))
  );
  homomorph_possible(FromGraph,ToGraph) == (
     card(vertices(FromGraph'EDGES)) <= card(vertices(ToGraph'EDGES)) &
     card((FromGraph'EDGES)) <= card((ToGraph'EDGES)) &
     ran(FromGraph'LABEL) = ran(ToGraph'LABEL) 
    );

  /* The (most general, i.e. ternary) definition of isomorphism*/
  isomorphic1(FirstGraph, iso, SecondGraph) == (  
    LET V1,V2 BE
      V1 = vertices(FirstGraph'EDGES) &
      V2 = vertices(SecondGraph'EDGES) 
    IN
		iso : V1 >->> V2 &
		!x.( x:V1 => FirstGraph'LABEL(x) = SecondGraph'LABEL(iso(x))) &
	
		// derived constraint for efficiency:
		//!l.(l:ran(SecondGraph'LABEL) => card({x|x|->l : FirstGraph'LABEL}) = card({y|y:Vertices & y|->l:SecondGraph'LABEL})) &
	
		!(x,y).( x|->y: FirstGraph'EDGES
			=> iso(x)|->iso(y) : SecondGraph'EDGES) &
		!(x,y).( x|->y: SecondGraph'EDGES
			=> iso~(x)|->iso~(y) : FirstGraph'EDGES)
    END
  );
  
  vertices(EdgeRelation) == (dom(EdgeRelation)  \/ ran(EdgeRelation));
  
  ValidPattern(pat) == (
       pat : (struct(LABEL:Vertices<->Labels, EDGES:Vertices<->Vertices)) &
       pat'EDGES <: Template'EDGES &
       pat'LABEL = Template'LABEL & // we fix the labeling
        IF MinMatchPOS=1 THEN
          #p.(p:ran(graphs) & p'SIGN="POS" & #isop.(homomorph_with(pat, isop, p))) 
        ELSE 
       
          // what we need is to reify existential quantifier #isop to be able to reify card !
          card({p|p:ran(graphs) & p'SIGN="POS" & homomorph_possible(pat, p)}) >= MinMatchPOS &
          card({p|p:ran(graphs) & p'SIGN="POS" & #isop.(homomorph_with(pat, isop, p))}) >= MinMatchPOS
        END 
     )

VARIABLES
  Patterns
INVARIANT
  Patterns : seq(struct(LABEL:Vertices+->Labels, EDGES:Vertices<->Vertices))
INITIALISATION
  //Patterns := {rec(LABEL:{(x1,a),(x2,b),(x3,a),(x4,a),(x5,a),(x6,a),(x7,a),(x8,a)}, EDGES:{(x1,x2),(x2,x3)})}
  Patterns := {}
OPERATIONS
   AddNewPattern(n,pat) = SELECT n=1+card(Patterns) & n<=MaxPatternsToFind & 
                                 ValidPattern(pat) &
                                !(p1).(p1:ran(Patterns) =>
                                      (p1 /= pat
                                         & not (#iso.(isomorphic1(p1, iso, pat)))
                                      ))
                        THEN
        Patterns := Patterns \/ {n|->pat}
   END
END
\end{lstlisting}

