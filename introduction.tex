% !TEX root = notes.tex

\section{Introduction}
Many real world problems present themselves as varying combinations of multiple smaller problems. 
These types of problems lend themselves for a declarative approach as knowledge representation offers a natural model satisfying `The Principle of Elaboration Tolerance'~\citep{elaboration_tolerance}: 
declarative specifications are easily adapted to new requirements or changed circumstances.
Conversely, the smaller problems in these composite structures are often already NP or coNP-complete.
Combining these already complex problems often raises the computational complexity of the composite problem, up to a level where it cannot be expressed using first order logic.
These problems become higher order logic problems: We study the \emph{Graph Mining} problem as an example.

Specification languages offer varying levels of support for higher order logic.
On the one hand, meta-programming, as known from Logic Programming~\citep{abramson1989meta}, has inspired the introduction of higher-order atoms in DLVHex~\citep{conf/ijcai/EiterIST05}  and the higher-order syntax in HiLog~\citep{chen1993hilog}.
As in Prolog, predicate symbols can be either constants (first order case) or variables (second order case). The latter range over predicate names, and not the predicate space itself, essentially combining second order syntax with first order semantics. 
%This cannot model the graph mining problem.

On the other hand, formal specification languages such as Z \citep{Bowen:Z}, B \citep{Abrial:BBook}, Event-B~\citep{Abrial10} and TLA~\citep{books/aw/Lamport2002}
 extend predicate logic with set theory and offer higher order datastructures. ProB~\citep{journals/sttt/LeuschelB08} is a constraint solver, animator and model checker for such languages, implemented in SICStus Prolog.
We can express the graph mining problem in ProB directly using higher order logic but, as we show later, these systems miss flexibility and the performance of revolutionary techniques such as CDCL~\citep{DBLP:conf/iccad/SilvaS96}.

Therefore, we also look at specification languages that do not allow higher order syntax.
Examples of such languages are the IDP~\citep{WarrenBook/DeCatBBD16} and the ASP~\citep{conf/rweb/EiterIK09} language.
For these languages, several techniques exist that allow the user to simulate higher order logic to model problems such as graph mining, potentially offering better performance than systems that allow higher order logic directly.

%We study the \emph{Graph Mining} problem as an example problem of such a raise in complexity, and detail some of the techniques that can be used to handle this.
%This includes the generalization of ad-hoc techniques as well as the combination of techniques from different logics.

%Graph mining is a specific kind of \emph{frequent pattern mining}, 
%%which is one of the ``super-problems'' in Data Mining.~\citep{?}
%%In its basic form it is }
%the task of enumerating patterns which occur frequently in a dataset.
%A first class of \emph{pattern mining} is \emph{unstructured mining}, such as \emph{itemset mining}, where the pattern is a set of items without any additional structural relation between the different items. 
%Because it does not require any structure, the \emph{unstructured mining} task does not compose multiple smaller problems.
%Therefore the rise in computational complexity associated with problem composition does not occur and the problem is even of propositional nature.
%\citet{tias_original} have demonstrated how these mining problems can successfully be modeled in a declarative way using CP techniques.
%Itemset mining modeling using ASP was studied in \citet{asp_itemset}.

Graph mining is a specific kind of \emph{frequent pattern mining}, the task of enumerating patterns which occur frequently in a dataset.
For graph mining, complexe structures such as edge relations take the role of patterns. The related task of itemset mining, where patterns exhibit no structure, is propositional in nature and has been modeled by \citet{tias_original} using CP and \citet{asp_itemset} using ASP.
Having additional structure in patterns complicates matters as e.g. checking for duplicity turns into the  \NP-complete problem of graph homomorphism~\citep{Lev73}, and its many variations, which in imperative languages lead to many different algorithms~\citep{gspan,theta_subsumption}.
A declarative approach can express these variations with only minimal changes.
%leading to the concept of homomorphism checks.
%\Likewise, \matthias{(also due to the additional structure)}, we need to represent the objects in the graph mining problem in a way that makes it non-obvious to say whether two objects are identical, which gives rise to subgraph isomorphism detection.
%Many different interpretations can be given to 
%these two problems
%\matthias{homomorphism checks and isomorphism detection}~
%this problem\citep{subtree_overview}, and in general these variations of the homomorphism 
%\st{and isomorphism concepts }
%concept
%are NP-complete~\citep{Lev73}.
%While this gives rise to many wildly different algorithms in imperative languages~\citep{gspan,theta_subsumption}, these variations can be specified with only minimal changes using a declarative modelling approach.

In our case study of the graph mining problem, we start with from the mathematical model of graph mining, which is inherently higher order, and identify the following contributions:
\begin{compactitem}
\item We identify the higher order aspects of the graph mining problem and show how the problem can be modeled in IDP, ASP and ProB, proposing concrete modeling techniques.
% (which supports Existential Second Order and inductive definitions)
%in ASP ($\Sigma_{2}^{p}$), and in ProB (which supports higher order sets), proposing concrete modeling techniques such as the disjoint union technique. 
We also identify a set of desirable properties for a declarative encoding of the graph mining problem.
%We compare how the different solutions perform, both with respect to the desirable properties as to computation time.
%\item We propose higher order language constructs for IDP and ProB extensions which greatly improve the ease of expressing the graph mining and other higher order problems.
%\item We indicate how to shift the burden of addressing the higher order aspects of the problem from modeler to the solver. 
\item 
We propose a higher order encoding that closely follows the mathematical model of graph mining, and satisfies all desirable properties of a declarative graph mining model.
%We indicate how to shift the burden of addressing the higher order aspects of the problem using encoding techniques from modeler to the solver. 
%Concretely, solver support is needed to translate the proposed higher order language constructs from the previous item to first order using e.g the disjoint union modeling technique.
We indicate how additional solver support can exploit the additional structure in this encoding to work more efficiently.
%these expressive constructs to work more efficient.
\end{compactitem}
The paper is structured as follows: Section \ref{sec:formalization} introduces graph mining formally, Section \ref{sec:modeling} discusses the how to model the problem in IDP, ASP and ProB, identifying a set of desirable properties.
Then, Section~\ref{sec:performance} discusses the performance of these systems.
Section \ref{sec:extension} discusses a faithful encoding of the graph mining problem in an KR language enriched with HO, and its possible solver implementation. Section \ref{sec:conclusion} draws conclusions and outlines possible future research directions.
