\section{Code in ProB, IDP, ASP}
\matthias{This is the working ProB code that we'll show as an example}
\VerbatimInput{original_prob_files/PositiveAndNegative.mch}
\pagebreak

\matthias{This is the positive constraint expressed in idp}

\VerbatimInput[label=The main merged theory]{IDPencoding/Merged.idp}
\VerbatimInput[label=IDP positive encoding]{IDPencoding/PositiveInfo.idp}
\VerbatimInput[label=IDP negative encoding]{IDPencoding/NegativeInfo.idp}

\lstset{basicstyle=\footnotesize\ttfamily,breaklines=true}
\begin{lstlisting}[caption=ASP positive matching, style=model]
positive_match(G) | not_positive_match(G) :- positive(G).

1 { map(G,X,V) : node(G,V) } 1 :- positive(G), invar(X).

:- positive_match(G), map(G,X,V1), map(G,Y,V2), t_edge(X,Y), 
not edge(G,V1,V2), invar(X), invar(Y).

positive_count(N) :- N = #count{G:positive_match(G)}.

:- positive_count(N), N < 2.
\end{lstlisting}

\begin{lstlisting}[caption=ASP negative matching, style=model]
%Saturated Representation

%negative constraints to check not matching negative graphs

map(G,X,v1) | map(G,X,v2) | map(G,X,v3) | map(G,X,v4) :- invar(X), negative(G).

map(G,X,V) :- saturated(G), t_node(X), node(G,V).

saturated(G) :- t_edge(X,Y), map(G,X,V1), map(G,Y,V2), not edge(G,V1,V2), negative(G), invar(X), invar(Y).
saturated(G) :- map(G,X,V),  map(G,Y,V), X != Y, invar(X), invar(Y). // we cannot map two different template nodes to the same 

negative_match(G) :- not saturated(G), negative(G).

negative_count(N) :- N = #count{G:negative_match(G)}.

:- negative_count(N), N > 1.

\end{lstlisting}

\begin{lstlisting}[caption=Canonicity template-based check, style=model]
iso(X,x1) | iso(X,x2) | iso(X,x3) | iso(X,x4) :- invar(X).

candidate_var(X) :- iso(_,X).

%not iso!
iso_saturated :- invar(X1), invar(X2), iso(X1,V1), iso(X2,V2),     t_edge(V1,V2), not t_edge(X1,X2). 
iso_saturated :- invar(X1), invar(X2), iso(X1,V1), iso(X2,V2), not t_edge(V1,V2),     t_edge(X1,X2).

iso(X,V) :- invar(X), t_node(V), iso_saturated.

d1(X) :-     invar(X), not candidate_var(X). 
d2(X) :- not invar(X),     candidate_var(X).

not_equal :- d1(X). % check that in fact candidate is different from the pattern itself
not_equal :- d2(X). % check that in fact candidate is different from the pattern itself

iso_saturated :- not not_equal. % should not be completely equal

min_d1(N) :- N = #min{ X: d1(X) }, not iso_saturated.
min_d2(N) :- N = #min{ X: d2(X) }, not iso_saturated.

iso_saturated :- min_d1(N1), min_d2(N2), N1 > N2.
\end{lstlisting}

\begin{lstlisting}[caption=Auxilary predicates -- probably should be moved to appendix, style=model]
%selects subpattern

t_path(X,Y) :- t_edge(X,Y), invar(X), invar(Y).
t_path(X,Y) :- t_edge(X,Z), t_path(Z,Y), invar(X).

:- invar(X), invar(Y), not t_path(X,Y).

0 { invar(X) } 1 :- t_node(X).
% auxilary constraints


edge(G,Y,X) :- edge(G,X,Y).
t_edge(Y,X) :- t_edge(X,Y).
node(G,Y)   :- edge(G,Y,_).
t_node(X)   :- t_edge(X,_).
\end{lstlisting}

\begin{lstlisting}[caption=Canonicity previous solution isomorphism check, style=model]
iso(s1,X,x1) | iso(s1,X,x2) :- invar(X).
iso(s2,X,x2) | iso(s2,X,x3) :- invar(X).

candidate_var(G,X) :- iso(G,_,X).

iso_saturated(G) :- invar(X1), invar(X2), iso(G,X1,V1), iso(G,X2,V2),     t_edge(V1,V2), not t_edge(X1,X2). 
iso_saturated(G) :- invar(X1), invar(X2), iso(G,X1,V1), iso(G,X2,V2), not t_edge(V1,V2),     t_edge(X1,X2). 
iso_saturatea(G) :- not equal(G), iso(G,_,_). 

iso(G,X,V) :- invar(X), t_node(V), iso_saturated(G).

:- not iso_saturated(G), iso(G,_,_).

d1(G,X) :-     invar(X), not candidate_var(G,X), iso(G,_,_).
d2(G,X) :- not invar(X),     candidate_var(G,X).

not_equal(G) :- d1(G,X). % check that in fact candidate is different from the pattern itself
not_equal(G) :- d2(G,X). % check that in fact candidate is different from the pattern itself

equal(G) :- not not_equal(G), iso(G,_,_).
\end{lstlisting}

